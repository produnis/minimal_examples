% Options for packages loaded elsewhere
\PassOptionsToPackage{unicode}{hyperref}
\PassOptionsToPackage{hyphens}{url}
\PassOptionsToPackage{dvipsnames,svgnames,x11names}{xcolor}
%
\documentclass[
  letterpaper,
  DIV=11,
  numbers=noendperiod]{scrartcl}

\usepackage{amsmath,amssymb}
\usepackage{iftex}
\ifPDFTeX
  \usepackage[T1]{fontenc}
  \usepackage[utf8]{inputenc}
  \usepackage{textcomp} % provide euro and other symbols
\else % if luatex or xetex
  \usepackage{unicode-math}
  \defaultfontfeatures{Scale=MatchLowercase}
  \defaultfontfeatures[\rmfamily]{Ligatures=TeX,Scale=1}
\fi
\usepackage{lmodern}
\ifPDFTeX\else  
    % xetex/luatex font selection
\fi
% Use upquote if available, for straight quotes in verbatim environments
\IfFileExists{upquote.sty}{\usepackage{upquote}}{}
\IfFileExists{microtype.sty}{% use microtype if available
  \usepackage[]{microtype}
  \UseMicrotypeSet[protrusion]{basicmath} % disable protrusion for tt fonts
}{}
\makeatletter
\@ifundefined{KOMAClassName}{% if non-KOMA class
  \IfFileExists{parskip.sty}{%
    \usepackage{parskip}
  }{% else
    \setlength{\parindent}{0pt}
    \setlength{\parskip}{6pt plus 2pt minus 1pt}}
}{% if KOMA class
  \KOMAoptions{parskip=half}}
\makeatother
\usepackage{xcolor}
\setlength{\emergencystretch}{3em} % prevent overfull lines
\setcounter{secnumdepth}{-\maxdimen} % remove section numbering
% Make \paragraph and \subparagraph free-standing
\makeatletter
\ifx\paragraph\undefined\else
  \let\oldparagraph\paragraph
  \renewcommand{\paragraph}{
    \@ifstar
      \xxxParagraphStar
      \xxxParagraphNoStar
  }
  \newcommand{\xxxParagraphStar}[1]{\oldparagraph*{#1}\mbox{}}
  \newcommand{\xxxParagraphNoStar}[1]{\oldparagraph{#1}\mbox{}}
\fi
\ifx\subparagraph\undefined\else
  \let\oldsubparagraph\subparagraph
  \renewcommand{\subparagraph}{
    \@ifstar
      \xxxSubParagraphStar
      \xxxSubParagraphNoStar
  }
  \newcommand{\xxxSubParagraphStar}[1]{\oldsubparagraph*{#1}\mbox{}}
  \newcommand{\xxxSubParagraphNoStar}[1]{\oldsubparagraph{#1}\mbox{}}
\fi
\makeatother


\providecommand{\tightlist}{%
  \setlength{\itemsep}{0pt}\setlength{\parskip}{0pt}}\usepackage{longtable,booktabs,array}
\usepackage{calc} % for calculating minipage widths
% Correct order of tables after \paragraph or \subparagraph
\usepackage{etoolbox}
\makeatletter
\patchcmd\longtable{\par}{\if@noskipsec\mbox{}\fi\par}{}{}
\makeatother
% Allow footnotes in longtable head/foot
\IfFileExists{footnotehyper.sty}{\usepackage{footnotehyper}}{\usepackage{footnote}}
\makesavenoteenv{longtable}
\usepackage{graphicx}
\makeatletter
\def\maxwidth{\ifdim\Gin@nat@width>\linewidth\linewidth\else\Gin@nat@width\fi}
\def\maxheight{\ifdim\Gin@nat@height>\textheight\textheight\else\Gin@nat@height\fi}
\makeatother
% Scale images if necessary, so that they will not overflow the page
% margins by default, and it is still possible to overwrite the defaults
% using explicit options in \includegraphics[width, height, ...]{}
\setkeys{Gin}{width=\maxwidth,height=\maxheight,keepaspectratio}
% Set default figure placement to htbp
\makeatletter
\def\fps@figure{htbp}
\makeatother

\KOMAoption{captions}{tableheading}
\makeatletter
\@ifpackageloaded{tcolorbox}{}{\usepackage[skins,breakable]{tcolorbox}}
\@ifpackageloaded{fontawesome5}{}{\usepackage{fontawesome5}}
\definecolor{quarto-callout-color}{HTML}{909090}
\definecolor{quarto-callout-note-color}{HTML}{0758E5}
\definecolor{quarto-callout-important-color}{HTML}{CC1914}
\definecolor{quarto-callout-warning-color}{HTML}{EB9113}
\definecolor{quarto-callout-tip-color}{HTML}{00A047}
\definecolor{quarto-callout-caution-color}{HTML}{FC5300}
\definecolor{quarto-callout-color-frame}{HTML}{acacac}
\definecolor{quarto-callout-note-color-frame}{HTML}{4582ec}
\definecolor{quarto-callout-important-color-frame}{HTML}{d9534f}
\definecolor{quarto-callout-warning-color-frame}{HTML}{f0ad4e}
\definecolor{quarto-callout-tip-color-frame}{HTML}{02b875}
\definecolor{quarto-callout-caution-color-frame}{HTML}{fd7e14}
\makeatother
\makeatletter
\@ifpackageloaded{caption}{}{\usepackage{caption}}
\AtBeginDocument{%
\ifdefined\contentsname
  \renewcommand*\contentsname{Table of contents}
\else
  \newcommand\contentsname{Table of contents}
\fi
\ifdefined\listfigurename
  \renewcommand*\listfigurename{List of Figures}
\else
  \newcommand\listfigurename{List of Figures}
\fi
\ifdefined\listtablename
  \renewcommand*\listtablename{List of Tables}
\else
  \newcommand\listtablename{List of Tables}
\fi
\ifdefined\figurename
  \renewcommand*\figurename{Figure}
\else
  \newcommand\figurename{Figure}
\fi
\ifdefined\tablename
  \renewcommand*\tablename{Table}
\else
  \newcommand\tablename{Table}
\fi
}
\@ifpackageloaded{float}{}{\usepackage{float}}
\floatstyle{ruled}
\@ifundefined{c@chapter}{\newfloat{codelisting}{h}{lop}}{\newfloat{codelisting}{h}{lop}[chapter]}
\floatname{codelisting}{Listing}
\newcommand*\listoflistings{\listof{codelisting}{List of Listings}}
\makeatother
\makeatletter
\makeatother
\makeatletter
\@ifpackageloaded{caption}{}{\usepackage{caption}}
\@ifpackageloaded{subcaption}{}{\usepackage{subcaption}}
\makeatother

\ifLuaTeX
  \usepackage{selnolig}  % disable illegal ligatures
\fi
\usepackage{bookmark}

\IfFileExists{xurl.sty}{\usepackage{xurl}}{} % add URL line breaks if available
\urlstyle{same} % disable monospaced font for URLs
\hypersetup{
  pdftitle={Test Long list},
  pdfauthor={me},
  colorlinks=true,
  linkcolor={blue},
  filecolor={Maroon},
  citecolor={Blue},
  urlcolor={Blue},
  pdfcreator={LaTeX via pandoc}}


\title{Test Long list}
\author{me}
\date{}

\begin{document}
\maketitle


\begin{tcolorbox}[enhanced jigsaw, leftrule=.75mm, rightrule=.15mm, bottomrule=.15mm, colback=white, toprule=.15mm, title=\textcolor{quarto-callout-note-color}{\faInfo}\hspace{0.5em}{Note}, left=2mm, bottomtitle=1mm, opacitybacktitle=0.6, colbacktitle=quarto-callout-note-color!10!white, colframe=quarto-callout-note-color-frame, toptitle=1mm, breakable, titlerule=0mm, opacityback=0, arc=.35mm, coltitle=black]

test

\end{tcolorbox}

\begin{tcolorbox}[enhanced jigsaw, rightrule=.15mm, breakable, arc=.35mm, leftrule=.75mm, bottomrule=.15mm, colframe=quarto-callout-note-color-frame, colback=white, opacityback=0, left=2mm, toprule=.15mm]
%\begin{minipage}[t]{5.5mm}
%\textcolor{quarto-callout-note-color}{\faInfo}
%\end{minipage}%

      

\textcolor{quarto-callout-note-color}{\faInfo}~ Der Datensatz
\texttt{Neugeborene.sav} (verfügbar unter
\url{https://www.produnis.de/tabletrainer/data/Neugeborene.sav})
beinhaltet Daten von 120 Neugeborenen und ihrer Eltern.

FOO und BAAR

\begin{enumerate}
\def\labelenumi{\alph{enumi})}
\item
  Laden Sie den SPSS-Datensatz \texttt{Neugeborene.sav} in Ihre
  \texttt{R}-Session und überführen Sie ihn in eine \texttt{data.table}
  mit dem Namen \texttt{ng2}.
\item
  In einigen Variablen finden Sie die Merkmalsausprägungen 9, 99 oder
  999. Diese stehen für fehlende Werte und müssen in \texttt{NA}
  umgewandelt werden. Somit ist sichergestellt, dass \texttt{R} diese
  Werte in weiteren Auswertungen nicht berücksichtigt.
\item
  Wandeln Sie die Variable \texttt{SEX} in einen Factor mit den Levels
  ``\texttt{männlich}'' (statt 1) und ``\texttt{weiblich}'' (statt 2)
  um.
\item
  Bilden Sie aus der Variable Geburtsgewicht (\texttt{GEBGEWI}) eine
  neue Variable (\texttt{GEWIKAT}), welche das Geburtsgewicht den
  folgenden Kategorien zuordnet:

  \begin{itemize}
  \tightlist
  \item
    \(\le\) 2500g\\
  \item
    \(>\) 2500 bis 3000g
  \item
    \(>\) 3000 bis 3500g
  \item
    \(>\) 3500 bis 4000g
  \item
    \(>\) 4000g
  \end{itemize}
\item
  Berechnen Sie zur Variable Geburtsgewicht folgende
  Stichprobenmerkmale:

  \begin{itemize}
  \tightlist
  \item
    Minimum, 5. Perzentil, 1. Quartil, Median, Mittelwert,
  \item
    \begin{enumerate}
    \def\labelenumii{\arabic{enumii}.}
    \setcounter{enumii}{2}
    \tightlist
    \item
      Quartil, 95. Perzentil, Maximum, Interquartilabstand
    \end{enumerate}
  \end{itemize}

  getrennt für Jungen und Mädchen.
\item
  Erstellen Sie Boxplots des Geburtsgewichts für alle Kinder, sowie
  separat für Jungen und Mädchen.
\item
  Erstellen Sie zur Variable \texttt{GEWIKAT} je eine Häufigkeitstabelle
  und ein Säulendiagramm für a) die gesamte Stichprobe und b) unter
  Berücksichtigung des 2. Merkmals \texttt{SEX}
\item
  Analysieren Sie den (linearen) Zusammenhang zwischen dem
  Geburtsgewicht {[}\texttt{GEBGEWI}{]} und der Körpergröße bei Geburt
  {[}\texttt{GEBGROE}{]}.
\item
  Einfluss des BMI

  \begin{itemize}
  \tightlist
  \item
    Bilden Sie aus den Variablen Größe des Vaters {[}\texttt{VATGROE}{]}
    und Gewicht des Vaters {[}\texttt{VATGEW}{]} den Body Mass Index
    {[}\texttt{VATBMI}{]} (kg/m\textsuperscript{2}).
  \item
    Bilden Sie den BMI der Mutter {[}\texttt{MUTBMI}{]} aus den
    Variablen Gewicht der Mutter {[}\texttt{MUTGEW}{]} und Größe der
    Mutter {[}\texttt{MUTGROE}{]}.
  \item
    Gibt es einen (linearen) Zusammenhang zwischen dem BMI der Mutter
    und dem des Vaters?
  \item
    Hat der BMI der Mutter einen Einfluss auf das Geburtsgewicht des
    Neugeborenen?
  \end{itemize}
\item
  Bilden Sie aus der Variable Gewicht im Alter von 6 Wochen
  {[}\texttt{FUGEW}{]} und Größe im Alter von 6 Wochen
  {[}\texttt{FUGROE}{]} die Variable \emph{Ponderal Index im Alter von 6
  Wochen}. \newline Für Säuglinge lautet die Formel
  \(PI = 100 \cdot \frac{g}{cm^3}\).
\item
  Bilden Sie eine neue Variable: Gewichtszunahme des Kindes von Geburt
  bis zum Alter von 6 Wochen.
\item
  Bilden Sie eine neue Variable: Gewichtszunahme von Geburt bis zum
  Alter von 6 Wochen in \% vom Geburtsgewicht.
\item
  Wie viele Kinder wurden gestillt ({[}\texttt{JSTILL}{]}, (1,2))?
\item
  Vergleichen Sie die gestillten und die nicht gestillten Kinder

  \begin{itemize}
  \tightlist
  \item
    bezüglich ihres Gewichts im Alter von 6 Wochen,
  \item
    ihrer Gewichtszunahme (Geburt -- 6 Wochen),
  \item
    ihrer prozentualen Gewichtszunahme (Geburt -- 6 Wochen),
  \item
    ihres Ponderal Index im Alter von 6 Wochen.
  \end{itemize}
\item
  Bilden Sie eine neue Variable Schwangerschaftsdauer
  {[}\texttt{SCHDAUG}{]} in Gesamttagen, die Sie aus den Variablen
  Schwangerschaftsdauer in (ganzen) Wochen (\texttt{SCHDAUW}, fehlende
  Werte =99) und Schwangerschaftsrestdauer in Tagen (\texttt{SCHDAUT};
  fehlende Werte=9; ``.'' = 0) bilden.\newline Hat die
  Schwangerschaftsdauer einen Einfluss auf das Geburtsgewicht?
\item
  Bilden Sie aus der Variable Nationalität der Mutter
  {[}\texttt{NATMUT}{]} eine neue Variable, welche die Nationalität der
  Mutter in 3 Kategorien zusammenfasst: \texttt{deutsch} (NATMUT=D),
  \texttt{türkisch} (NATMUT=TR) und \texttt{sonstige} (alle anderen,
  auch die ohne Angabe).
\item
  Unterscheiden sich die Kinder von Müttern der verschiedenen
  Nationalitäten hinsichtlich ihres Geburtsgewichts und ihres Ponderal
  Index im Alter von 6 Wochen?
\item
  Werden die Kinder von Müttern unterschiedlicher Nationalitäten gleich
  häufig gestillt?
\item
  Vergleichen Sie das mittlere Geburtsgewicht mit der Referenz 3500g
  (t-Test für eine Stichprobe).
\item
  Vergleichen Sie das mittlere Geburtsgewicht von männlichen und
  weiblichen Neugeborenen (t-Test für zwei Stichprobe).
\end{enumerate}

\end{tcolorbox}
      

guten tach




\end{document}
